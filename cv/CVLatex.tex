\documentclass{article}
\usepackage{parskip} %gives 1.5 spacing
\usepackage{graphicx,rotating}
\usepackage[colorlinks=true,urlcolor=blue]{hyperref}
\usepackage[scale=0.75]{geometry} %adjust page margins

\newcommand{\HRule}{\rule{\linewidth}{0.5mm}}
\begin{document}

\begin{center} 

\textbf{\huge{DENNIS LIU}}
\\[0.7cm]

tdennisliu@gmail.com \\
\url{www.linkedin.com/in/dennis-liu-5037} \\
\url{tdennisliu.github.io}

\end{center}
\emph{To enact positive change and understand the greater world community} 
%\HRule


\HRule

{\footnotesize EDUCATION}
\\

\textbf{\large University of Adelaide} 

\emph{Ph.D candidate: Mathematics (Statistics)} \hfill \texttt{2017 to present}\\[0.2cm]
\footnotesize{\emph{Data to Decisions CRC Big Data Scholarship}}\\

My research focuses on building frameworks to model social behaviours and infectious diseases using digital, non-traditional datasets. I am interested in analysing human behaviours and analysing their effects on models and forecasts of public health measures. I have recently been working with a wider group of researchers to provide forecasts of COVID-19 cases to the Australian Government and Department of Health. I am proficient in Python and standard quantitative packages (numpy, pandas, sklearn etc.), and have some familiarity with R and Matlab.

Key skills involved are:

\begin{itemize}
	\item Bayesian inference and machine learning
	\item Building and interrogating statistical models
	\item Event prediction
\end{itemize}

\textbf{\emph{Publications \& Conferences}}
\begin{table}[h!] \small
\begin{tabular}{ p{12.1cm} c r}

How to flatten the curve of coronavirus, a mathematician explains. \emph{The Conversation} & & \texttt{2020} \\
	
 Elucidating user behaviours in a digital health surveillance system to correct prevalence estimates. & & \texttt{Acceptance pending} \\

 Predicting influenza testing associated with family doctors by using a web-based surveillance system. & & \texttt{Acceptance pending} \\

 Complex Human Data Summer School & & \texttt{2018} \\
 Policy Relevant Infectious Disease Simulation and Mathematical Modelling& & \texttt{2018} \\

\end{tabular}
\end{table}


\emph{B. Ma. Comp. Sci (Applied)} \hfill  \texttt{2011 to 2015}

{\footnotesize
\begin{itemize}
\item Overloaded course work 2011 to 2012, studied part time while working full time 2014 to 2015
\end{itemize} }

\emph{B. Eng. (Chemical) with First Class Honours} \hfill \texttt{2009 to 2013}
\emph{Dean's Merit Certificate for Outstanding Academic Achievement} \hfill \texttt{2009 to 2012}
\\

%\textbf{\large Pulteney Grammar School} 

%\emph{SACE}\hfill \texttt{2004 to 2008}
%{\footnotesize
%\begin{itemize}
%\item Merits in Chemistry, Mathematical Studies and Specialist Mathematics
%\item TER of 99.85
%\item EE Scarfe Memorial Prize for Specialist Mathematics
%\item Participant in Australian Intermediate Mathematics Olympiad 

%\end{itemize}
%}

\HRule

{\footnotesize EMPLOYMENT EXPERIENCE}
\\

\textbf{\large Santos Ltd.}

\emph{Unit Controller Engineer} \hfill \texttt{2015 to 2017}

{\footnotesize
My previous employment required me to work in a remote location in a gas plant control room, where gas production from over 600 wells and multiple facilities is fed to the plant through compressors controlled by myself, to be processed all hours of the day.

\begin{itemize}
\item Communicate issues with other stakeholders clearly and concisely
\item Important commercial and safety decisions are regularly made independently, often employing critical problem solving skills to arrive to a solution
\end{itemize}
}
\emph{Graduate Facilities Engineer} \hfill \texttt{2014 to 2015}

{\footnotesize
As a facilities engineer, I was responsible for various facility assets producing oil and gas, optimising production. Projects were delivered in a timely and economic manner, with deliverables and risks communicated clearly to key stakeholders for approval.

\begin{itemize}
\item Optimisation of gas engines and compressors
\item Project management
\item Risk assessment \\
\end{itemize} }


\textbf{\large Globe Education Centre}

\emph{Mathematics and Science Tutor} \hfill \texttt{2009 to 2014}
{\footnotesize
\begin{itemize}
\item Tutored students from the ages of 5 to 18 years of age in mathematics and science
\item Supervised individuals and large classes
\end{itemize} }
\HRule


{\footnotesize KEY STRENGTHS}

\begin{itemize}
\item Clear communication skills in time constrained environments
\item Programming experience in Excel VBA, Matlab and Python
\item Project management
\item Experienced in working collaboratively and critical decision making
\end{itemize}

\HRule


{\footnotesize REFEREES}

\begin{center}



\begin{tabular}[H]{ r l}
\textbf{\large Lewis Mitchell} & \\[0.3cm]
Organisation: & The University of Adelaide \\
Position: & Senior Lecturer \\
Email: & lewis.mitchell@adelaide.edu.au \\[1cm]

\textbf{\large Glenn Lydyard} & \\[0.3cm]
Organisation: & Santos Ltd. \\
Position: & Team Leader Advanced Analytics \& Production Technologies \\
Email: & glenn.lydyard@santos.com \\[1cm]

\textbf{\large Sungkook Kwon} & \\[0.3cm]
Organisation: & Globe Education Centre \\
Position: & Director \\
Email: & skkwon@gecaustralia.com \\[1cm]

\end{tabular}

\end{center}

\end{document}

